\subsection{Clinical trial}

We conducted a randomised controlled clinical trial of neurofeedback therapy intervention for ADHD/ADD in adults. The trial's main research aim was to focus on the model and internal mechanics of neurofeedback learning, to elucidate the primary role of cortical self-regulation in neurofeedback.

\paragraph{Trial Registration}. The trial was registered with ISRCTN as "Computer Enabled Neuroplasticity Treatment (CENT)", ISRCTN13915109.


\subsubsection{Methods/Design}
The intervention consisted of neurofeedback treatment with waiting list control group. Treatment involved ~40 sessions, two-five sessions per week with five-seven units of training in each one hour session. Training involved either theta/beta or sensorimotor-rhythm protocols. The protocols were adapted by novel addition of an inverse training condition to promote self-regulation. Follow-up will consist of self-report and executive function tests.

The individual alpha peak frequency (IAPF) of each participant was estimated from band power analysis of eye-opened and eye-closed baseline conditions, following \cite{Lansbergen2011}. The boundaries of each EEG frequency band for each participant are defined with respect to IAPF, e.g. theta is IAPF$\times$0.4 to IAPF$\times$0.6.

After randomisation between treatment and control groups, we assigned participants in the NFB treatment group to either TB or SMR training based on their IAPF-adjusted theta/beta ratio. Those with theta/beta ratio \textgreater 1 (n = 9) received reinforcement for simultaneous increase in beta and decrease in theta (over power estimated from per-session baseline) at electrode Fz. The rest (n = 16) got reinforcement for increase in SMR and decrease in theta at electrode C4. Band powers within the NFB protocols are adjusted by IAPF. 

NFB interventions were standardised by scheduling of the training sessions: session duration was fixed; and training blocks per session, sessions per week, timing of the break from training, and total duration of training were all constrained to equalise the intervention. Treatment group participants began their treatment by being briefed about all aspects of the NFB protocols, e.g. length, frequency, purpose. Outcome measures were taken when all participants in the treatment group had completed 40 sessions NFB.


\paragraph{Ethics}: Written informed consent for participation was obtained from all participants before entering the study. The protocol followed the Declaration of Helsinki for the rights of the participants and the procedures of the study. An ethical approval of the present research protocol for all participants was obtained from The Ethical Committee of the Hospital District of Helsinki and Uusimaa, 28/03/2012, 621/1999, 24 §. Participants were not remunerated.

\paragraph{Setting}: intake and outtake measurements were conducted at University of Helsinki campus. Treatment was administered at partner clinic Mental Capital Care Oy, Helsinki.

\paragraph{Randomisation}: Randomisation used a two-step procedure: randomisation of half of recruits, followed by adaptive allocation of the remainder to minimise baseline differences in prognostic variables.

\paragraph{Blinding}: Due to waiting list control design, trial was not blinded.

\paragraph{Participants}: 54 adult Finnish participants (29 females) were recruited after screening by psychiatric review. They had mean age 36 years (std.dev. 10 years), with 44 ADHD and 10 ADD diagnoses.

\paragraph{Measurements}: Participants’ symptoms were assessed by computerised attention test and self-report scales, at intake and outtake. Performance during neurofeedback trials was also measured.


\subsubsection{Results}
Participants were split between treatment and control groups. Following random group assignment, treatment began September 17, 2012. Of 54 inducted to the trial, eight dropped out from waiting list group, and two from treatment group.

\paragraph{Implementation}
In practice, NFB training consisted of ~40 sessions (range: 38-41) during two to four months. There was a mid-training pause of nominally two weeks. Patients came to the sessions two to five times a week. One session lasted ~1 hour, subdivided into self-report of mood, excitement, hours slept and hours awake; electrode attachment; baseline measurement; five to seven units of five minute NFB trials; and debrief including self-report of effort and frustration. During each session, patients played different NFB ‘game’ trials during which they got immediate visual reinforcement for classifier-matching states in their EEG. The scores per game trial are baseline-adjusted and averaged per session to form characteristic LCs. The content and purpose of each training session followed a phased timeline:

\begin{enumerate}
	\item Tutorial stage, for becoming accustomed to NFB, two practice sessions: participants were given normal NFB trials with baseline thresholds adjusted by a constant factor to make the training easier; 
	
	\item Beginner stage, for NFB training, 18 sessions up to halfway break: normal NFB with non-adjusted baseline thresholds; 
	
	\item Intermediate stage, for learning to self-regulate, ten sessions from half-way to session 30: normal training blocks were gradually reduced in number to half per session, and inverse training blocks introduced in their place; 
	
	\item Expert stage, for transfer training, ten sessions until session 40: as Intermediate stage, but also with one to two ‘transfer’ trials with no feedback stimuli.
	
\end{enumerate}


Initial analysis showed that, compared to waiting list control, neurofeedback promoted improvement of self-reported ADHD symptoms. Additional detailed analyses are ongoing.


\subsection{Use of the platform}
\todo[inline]{describe characteristics of using the platform, good e.g. flexibility, adding extra apps as desired, bad e.g. bluetooth streaming sucks}
