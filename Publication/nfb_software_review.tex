\subsection{Neurofeedback software}
Currently a large number of different NFB software packages exist, most of which are still actively used or still in development. The recent boom of wearable biosensors (such as cheap, commercial EEG devices like the Muse and Melon headbands) has also boosted the number of available personal NFB applications. Despite the popularity, very little literature exists reviewing NFB platforms. One report estimates the usefulness of various BCI frameworks for conducting NFB, and lists design considerations for such a system  \cite{huster2014brain}. In this section we attempt to comprehensively cover different types of software packages are available for NFB.

All available NFB solutions share three basic characteristics which are: 1. a method of interfacing with an EEG device, 2. capability to process the acquired EEG data in real-time and 3. the ability to generate feedback for the user. Outside these parameters different software packages can have vastly different properties. For instance hardware devices supported, licensing, and intended usage all vary greatly between different NFB solutions. The NFB software can roughly be divided into two categories: clinical and non-clinical. The clinical category contains software that is solely intended for various NFB therapies (ADHD therapy being the most common). The other software packages are aimed more for personal cognitive neuromodulation and entrainment (such as mediation and stress management). We have compiled a list (table \ref{nfbsoftware}) which, to our knowledge, contains all of the currently available software packages intended for NFB. 

Table \ref{nfbsoftware} lists 33 NFB software, alongside their respective licenses and other information. 
The 'License' column indicates which license was used when the software was published. In some cases a software package was released with the source code but without a specified license, as noted by a question mark. 
The 'Merit' column refers to scientific merit, defined here as the highest ranking publication the corresponding software was used in. The rankings were extracted from the Finnish Publication Forum, which is a nationally-accepted quality assessment forum for publication channels. Rank values for publication venues are 0 (no rank assigned  by the forum), 1 (basic), 2 (leading), and 3 (highest) \footnote{For more information see \url{http://www.julkaisufoorumi.fi/en/publication-forum}}. The merit value indicates whether the software has been used in NFB related research. For instance the BioExplorer software has been used to study the increase local gamma and beta band activity through NFB \cite{keizer2010enhancing} and the EEGer4 to study the effect of music on alpha/theta NFB \cite{gruzelier2014replication}.

The 'Last Update' column indicates the latest known time of publication for either software or documentation updates. This column indicates whether or not a project is still in active development. Twenty of the software packages are still clearly active (updates less than one year old), and seven more have received updates in the last three years, but the remaining six projects have been dormant for between four and twelve years.

Finally, in the last column the use case for each software (clinical vs non-clinical) is listed. The use case was determined on the basis of the developers' own descriptions from the web-page of the software.

\begin{table}[ht]
\centering
\begin{tabular}{lrrrc}
System & License & Merit & Last Update & Clinical use \\       
\hline
BioEra & Prop. & 1 & 2015-06-22 & No \\   
BrainBay & GPL & 1 & 2014-12-03 & No \\   
BrainAthlon & ? & 1 & 2004-01-01 & No \\   
EEGMIR & ? & 0 & 2003-12-30 & No \\   
ElectricGuru & ? & 0 & 2002-01-21 & No \\   
BioExplorer & Prop. & 2 & 2012-09-26 & No \\  
BioGraph Infiniti & Prop. & 1 & 2013-06-05 & Yes \\
BioTrace+ & Prop. & 2 & 2015-07-23 & Yes \\
BrainFeedbackPro & Prop. & 0 & 2015-11-19 & Yes \\
TruScan Neurofeedback & Prop. & 0 & 2015-11-19 & Yes \\
BrainMaster & Prop. & 1 & 2015-10-09 & Yes \\
BrainPaint & Prop. & 2 & 2012-01-01 & Yes \\
Cygnet & Prop. & 1 & 2015-11-01 & Yes \\
eBioo & Prop. & 0 & 2015-03-01 & Yes \\
EEGer4 & Prop. & 2 & 2013-06-10 & Yes \\
EventIDE & Prop. & 0 & 2015-08-18 & No \\
Mind Workstation & Prop. & 0 & 2011-08-31 & Yes \\
MindReflector & Prop. & 0 & 2013-09-19 & Yes \\
Neurofield & Prop. & 1 & 2015-02-06 & Yes \\
neuromore Studio & Prop. & 0 & 2015-11-06 & No \\
NeurOptimal & Prop. & 0 & 2015-07-01 & Yes \\
NeuroRT & Prop. & 0 & 2015-11-04 & Yes \\
OpenViBE & AGPL & 1 & 2015-10-02 & No \\
SmartMind3 & Prop. & 1 & 2015-01-01 & Yes \\
Melon - Brain Training & Prop. & 0 & 2015-02-28 & No \\
Muse App & Prop. & 0 & 2015-12-16 & No \\
Neurosurfer & Prop. & 0 & 2015-02-15 & Yes \\
BrainWaveOSC & ? & 0 & 2014-07-30 & No \\
OpenNFB & GPL & 0 & 2015-11-19 & No \\
WaveTuner & ? & 0 & 2013-10-16 & No \\
AlphaTrainer & ? & 0 & 2014-05-20 & No \\
Mindrun & MIT & 0 & 2015-09-29 & No \\
Resonanz & GPL & 0 & 2015-07-23 & No \\
\end{tabular}
    \caption{Currently available neurofeedback software packages}\label{nfbsoftware}
\end{table}


Table \ref{nfbsummary} summarises data on license (open-source vs. proprietary) and intended use (clinical vs. non-clinical). This illustrates that the majority of available software are commercial with  proprietary licenses. Although there are almost the same number of clinical and non-clinical software, all of clinical use software have a proprietary license. From this review, it is apparent that there is a clear lack of open-source solutions for clinical NFB. The closest option for such a platform would be the proprietary NeuroRT suite by Mensia Technologies, which like the CENT system, is built on top of the open-source OpenViBE platform. However, there is no NFB software for clinical use that is truly open-source from end-to-end, including both the signal processing back-end and the patient management front-end.


\begin{table}[h]
\centering
\begin{tabular}{lcc}
& Non-clinical  & Clinical \\       
Proprietary & 6 & 15 \\
Open-source  & 11 & 0 \\
\end{tabular}
    \caption{Division of use case and license in neurofeedback software packages}\label{nfbsummary}
\end{table}



\subsection{Wearable EEG sensors} % This section could go to discussion maybe?
In recent years there has been a sharp increase in the availability of ambulatory EEG sensors. This is partly due to technological advances, and also to the popularity of the quantified self movement. The quality of these devices varies from purely consumer-grade (also known as lifestyle applications) to more expensive but near laboratory-grade devices. The suitability of each device for NFB therapy must be tested, but the current trend looks promising as the devices are readily available and cheap compared to laboratory EEG. 

The standard software that usually accompanies these devices seems to be more oriented for self quantification and cognitive enhancement. For instance the MUSE band comes with an application that teaches meditation techniques. Devices like the MUSE, however, provide a communication protocol that allows other software to access the raw data of the device. Therefore, it is plausible that these devices can be used as an input to NFB software. The modification necessary would require that NFB software itself could be modified which further increases the need for open source solutions.
